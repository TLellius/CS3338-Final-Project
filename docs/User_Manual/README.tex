\documentclass{article}

\usepackage{geometry}
\usepackage{indentfirst}
\usepackage{fancyhdr}

\pagestyle{fancy}
\geometry{a4paper, margin=1in}

\begin{document}

\title{Read Me}
\author{Tony Lau}
\date{April 25th, 2025}

\fancyhf{}
\fancyhead[C]{README}
\fancyfoot[C]{\thepage}

\section{Jira Link (For Instructor): }
https://spring2025-cs3338-group-2.atlassian.net/jira/software/projects/VRMP/boards/6

\section{Objectives: }
The Viper Rocks! project plays a vital role in establishing a sustainable presence on the Moon: a vision that NASA shares. Mapping and analyzing the distribution of materials and resources on the Moon paves the way for further understanding of the Moon and its mysteries.

To better support this initiative, we wish for the project to involve the public. We hope to interest citizen scientists, both amateur and expert, to participate in learning and understanding lunar geology. With the use of our interactive tools to map and classify lunar rocks, we believe this will make lunar exploration an accessible and collaborative effort between the experts and the interested public!

\section{Access / Download Instructions: }
Currently, there seems to be no way access the Lunar Rocks! Website. It has conducted a beta test in early April, so it does not seem to be fully released yet. The following link has more information in regards to the progress of the project:
https://ascent.cysun.org/project/project/view/227

\end{document}